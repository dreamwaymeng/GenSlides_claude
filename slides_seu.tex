\documentclass[11pt,aspectratio=169]{beamer}

% Language mode: en (English)
\def\langmode{en}

% Theme colors - Original SEUBeamer template colors
\definecolor{themegreen}{RGB}{88,117,88}
\definecolor{themeblue}{RGB}{21,30,73}
\definecolor{themered}{RGB}{95,33,48}
\definecolor{themeyellow}{RGB}{253,208,0}

\colorlet{darkgray}{black!70}
\colorlet{stronghighlightcolor}{themered!150!black}
\colorlet{themecolorbg}{themegreen}
\colorlet{themecolorfg}{themeblue}

% Custom colors for quark diagrams
\definecolor{heavyblue}{RGB}{0,70,140}
\definecolor{lightquarkorange}{RGB}{230,120,50}

\usepackage{xcolor}
\usepackage{graphicx}
\usepackage[export]{adjustbox}
\usepackage{stackengine}
\usepackage{pifont}
\usepackage[framemethod=TikZ]{mdframed}
\usepackage{amsmath,amssymb}
\usepackage{array}
\usepackage{listings}
\usepackage{tcolorbox}
\usepackage{bm}
\usepackage{tabularx}
\usepackage{booktabs}
\usepackage{multirow}
\usepackage{ulem}
\usepackage{tikz}
\usetikzlibrary{arrows.meta,positioning,shapes,calc,decorations.pathmorphing}

% Beamer 模板设置
\setbeamertemplate{navigation symbols}{}
\setbeamertemplate{footline}[page number]
\setbeamertemplate{section in toc}[sections numbered]

% Itemize 格式设置
\setbeamertemplate{itemize item}[circle] % 项目符号设置为圆形
\setbeamertemplate{itemize subitem}{$\circ$} % 子项目符号设置为圆形
\setbeamertemplate{itemize subsubitem}[circle] % 子子项目符号设置为圆形

% Enumerate 格式设置
\setbeamertemplate{enumerate subitem}{\alph{enumii}.} % 子项目符号设置为小写字母
\setbeamertemplate{enumerate subsubitem}{\roman{enumiii}.} % 子子项目符号设置为罗马数字

% 颜色设置
\setbeamercolor{frametitle}{bg=, fg=themecolorfg}
\setbeamercolor{block title}{bg=themecolorbg!8, fg=themecolorfg}
\setbeamercolor{description item}{fg=themecolorfg}
\setbeamercolor{section in toc}{fg=themecolorfg, bg=}
\setbeamercolor{subsection in toc}{fg=black!80}
\setbeamercolor{caption name}{fg=themecolorfg}
\setbeamercolor{bibliography entry author}{fg=black}
\setbeamercolor{bibliography entry title}{fg=black}
\setbeamercolor{bibliography entry journal}{fg=black}
\setbeamercolor{bibliography entry note}{fg=black}
\setbeamercolor{bibliography item}{fg=themecolorfg}
\setbeamercolor{itemize item}{fg=themecolorbg} % 项目符号颜色设置
\setbeamercolor{itemize subitem}{fg=themecolorbg} % 子项目符号颜色设置
\setbeamercolor{itemize subsubitem}{fg=themecolorbg} % 子子项目符号颜色设置
\setbeamercolor{enumerate item}{fg=themecolorfg} % 枚举项颜色设置
\setbeamercolor{enumerate subitem}{fg=themecolorfg} % 枚举子项颜色设置
\setbeamercolor{enumerate subsubitem}{fg=themecolorfg} % 枚举子子项颜色设置

% ============================================
% 语言模式检测 (en 或 cn)
% ============================================
\newif\iflangcn
\ifdefined\langmode
  % 使用 \pdfstrcmp 或字符串比较 (更可靠)
  \makeatletter
  \def\@tempa{cn}
  \ifx\langmode\@tempa
    \langcntrue
  \else
    \langcnfalse
  \fi
  \makeatother
\else
  % 默认为中文模式
  \langcntrue
\fi

% ============================================
% 字体设置(根据语言模式)
% 定义全局字体变量,修改这里即可改变所有字体
% ============================================
\iflangcn
  % 中文模式:使用中文字体
  % 可选: \kaishu, \songti, \heiti, \fangsong, \yahei, \lishu 等
  \newcommand{\slidefont}{\kaishu}  % 标题字体 (frametitle, block title 等)
  \newcommand{\bodyfont}{\kaishu}     % 正文字体
    \AtBeginDocument{\bodyfont}

\else
  % 英文模式:使用默认字体
  \newcommand{\slidefont}{}  % 标题字体
  \newcommand{\bodyfont}{}  % 正文字体
\fi

% 字体设置 
\setbeamerfont{frametitle}{size=\Large,family=\slidefont}
\setbeamerfont{framesubtitle}{size=\normalsize,family=\slidefont}
\setbeamerfont{block title}{size=\large,family=\slidefont}
\setbeamerfont{section in toc}{size=\large,family=\slidefont}
\setbeamerfont{subsection in toc}{size=\normalsize,family=\slidefont}
\setbeamerfont{caption name}{size=\normalsize,family=\slidefont}
\setbeamerfont{caption}{size=\normalsize,family=\slidefont}
\setbeamerfont{footnote}{size=\tiny}
\setbeamerfont{description item}{size=\normalsize,family=\slidefont}

% 页脚模板设置
\setbeamertemplate{footline}{%
  \begin{beamercolorbox}[wd=\paperwidth, ht=2.25ex, dp=1ex, leftskip=0.5cm, rightskip=0.5cm]{title in head/foot}%
    \hfill{\textcolor{black!70}{\scriptsize \insertframenumber/\inserttotalframenumber}}\hspace*{1mm}\vspace*{2mm}
  \end{beamercolorbox}%
}

% 自定义标题模板 (使用 \slidefont 而不是 \bf)
\defbeamertemplate*{frametitle}{}[1][]{
  \nointerlineskip%
   \vspace*{-7mm}
  \hspace*{-1mm}
  \begin{beamercolorbox}[sep=0.3cm, wd=\paperwidth]{frametitle}
  \begin{center}
    {\bf\usebeamerfont{frametitle}\insertframetitle}
    {\usebeamerfont{framesubtitle}\color{black!80}\insertframesubtitle}
  \end{center}
       \vskip-4.5ex
    \hfill
    \raisebox{-3mm}{\includegraphics[width=20mm]{figs/logo_wb}}
           \vskip-1.5ex
    \begin{tikzpicture}[remember picture, overlay]
      \draw[line width=1pt, color=themecolorbg] (-14mm,0) -- (\paperwidth-28mm,0);
    \end{tikzpicture}
  \end{beamercolorbox}
  \vspace*{-7mm}
}



% 设置段落间距
\setlength{\parskip}{0.5em}

% 参考文献
\bibliography{reference.bib}
\normalem

% 数学字体主题设置
\usefonttheme[onlymath]{serif}

% 表格环境设置
\AtBeginEnvironment{table}{\setlength{\parskip}{0em}}


% ------------------

\newmdenv[
    hidealllines=true,
    leftline=true,
    linewidth=1pt,
    linecolor=darkgray,
    fontcolor=darkgray,
    innertopmargin=1pt,
    innerbottommargin=1pt,
    font=\bodyfont\normalsize,
]{myquote}

\newmdenv[
    hidealllines=true,
    linewidth=0pt,
    fontcolor=darkgray,
    font=\bodyfont\small,
]{myderive}

\newmdenv[
userdefinedwidth=0cm,
leftmargin=2cm,
linecolor=themecolorbg,
innerleftmargin=5pt,
 backgroundcolor=themecolorbg!10, % 设置填充颜色(此处为蓝色,20%透明度
align=center]{mybox}


\lstset{
    basicstyle=\ttfamily\color{stronghighlightcolor},
}

\lstdefinestyle{python}{
    language=Python,
    basicstyle=\normalsize\ttfamily,
    keywordstyle=\color{themecolorbg},
    stringstyle=\color{themered!150!black},
    commentstyle=\color{black!50},
    showstringspaces=false,
    backgroundcolor=\color{white},
    frame=single,
    framerule=0.25pt,
    framesep=3pt,
    rulecolor=\color{themecolorbg},
    numbers=left,
    numberstyle=\tiny\color{darkgray},
    numbersep=5pt,
    breaklines=true,
    escapeinside={(*@}{@*)}
}

\lstdefinestyle{latex}{
    language=[LaTeX]TeX,
    basicstyle=\normalsize\ttfamily,
    keywordstyle=\color{themecolorbg},
    stringstyle=\color{themered!150!black},
    commentstyle=\color{black!50},
    showstringspaces=false,
    backgroundcolor=\color{white},
    frame=single,
    framerule=0.25pt,
    framesep=3pt,
    rulecolor=\color{themecolorbg},
    numbers=left,
    numberstyle=\tiny\color{darkgray},
    numbersep=5pt,
    breaklines=true,
    escapeinside={(*@}{@*)}
}

% ------------------------

\newcommand{\makesection}[1][0.4]{
    \begin{frame}[plain]
        \centering
        \vspace*{7mm}
        \begin{tabular}{l}
            {\color{themecolorfg}
            \huge\slidefont
            \insertsectionnumber.\ \insertsection}\\
            \begin{tikzpicture}
                \fill[black!30] (0,0) rectangle (#1\textwidth, 1pt);
                \fill[themecolorfg] (0,0) rectangle (#1\textwidth/2, 1pt);
            \end{tikzpicture}
        \end{tabular}
    \end{frame}
}


\newcommand{\titlecolorbox}[2]{
    \hfill
\begin{tikzpicture}[remember picture, overlay]
  \node[
    anchor=north east,
    xshift=-0.5cm,   % 向左偏移
    yshift=-0.3cm    % 向下偏移
  ] at (current page.north east)
  {\includegraphics[width=3cm]{figs/logo_cl.pdf}};
\end{tikzpicture}


    \centering

     \vspace{5mm}
       \begin{tcolorbox}[colback=white!0, colframe=white!0,
    boxrule=0pt,
    toprule=0pt,
    leftrule=0pt,
    rightrule=0pt, arc=0mm,  width=\paperwidth,
   enlarge left by=-1in-\oddsidemargin,
    enlarge right by=-1in-\oddsidemargin ]
        \color{darkgray} \normalsize\slidefont
        #2
    \end{tcolorbox}
       \vspace{-5mm}
    \begin{tcolorbox}[colback=themecolorbg, colframe=themeyellow,
    boxrule=5pt,
    toprule=0pt,
    leftrule=0pt,
    rightrule=0pt, arc=0mm,  width=\paperwidth,
   enlarge left by=-1in-\oddsidemargin,
    enlarge right by=-1in-\oddsidemargin ]
        \color{white} \Huge\slidefont\centering
        #1
    \end{tcolorbox}
}

\newcommand{\highlight}[1]{\textcolor{themecolorbg}{#1}}
\newcommand{\ulhighlight}[1]{\textcolor{themecolorbg}{\color{themecolorbg}\uline{#1}}}
\newcommand{\stronghighlight}[1]{\textcolor{stronghighlightcolor}{#1}}
\newcommand{\ulstronghighlight}[1]{\textcolor{stronghighlightcolor}{\color{stronghighlightcolor}\uline{#1}}}


% Custom commands
\newcommand{\threebar}{\bar{3}_c}
\newcommand{\sixbar}{\bar{6}_c}

\begin{document}

\setbeamercovered{transparent}

%==============================================================================
% SLIDE 1: Title Slide
%==============================================================================
\begin{frame}[plain]
    \titlecolorbox{DeepQuark: Deep-Neural-Network Approach to Multiquark Bound States}{arXiv:2506.20555}

    \vspace{5mm}
    \begin{tabular}{>{\color{themecolorbg}}r@{\hspace{3pt}}l}
      \large Authors: & \large Wei-Lin Wu$^{1}$, \underline{Lu Meng}$^{2,3}$, Shi-Lin Zhu$^{1,4}$ \\[2mm]
      \small $^1$ & \small School of Physics, Peking University \\
      \small $^2$ & \small Institut f\"ur Theoretische Physik II, Ruhr-Universit\"at Bochum \\
      \small $^3$ & \small School of Physics, Southeast University \\
      \small $^4$ & \small Center of High Energy Physics, Peking University \\
    \end{tabular}

    \vspace{5mm}
    \centering
    Hadron Physics Workshop 2025
\end{frame}

\setcounter{framenumber}{0}

%==============================================================================
% Outline
%==============================================================================
\begin{frame}
    \frametitle{Outline}
    \setlength{\parskip}{0.2em}
    \tableofcontents
\end{frame}

%==============================================================================
% SECTION 1: INTRODUCTION
%==============================================================================
\section{Introduction}
\makesection

%==============================================================================
% SLIDE: Why Exotic Hadrons Matter
%==============================================================================
\begin{frame}{Why Exotic Hadrons Matter}
\begin{columns}[T]
\begin{column}{0.55\textwidth}
\textbf{\highlight{QCD: The Unsolved Theory}}
\begin{itemize}
    \item Strong force governs nuclear matter
    \item \stronghighlight{Confinement} remains poorly understood
    \item Quarks never observed in isolation --- why?
\end{itemize}

\vspace{0.4em}
\textbf{\highlight{Exotic Hadrons as Probes:}}
\begin{itemize}
    \item States beyond $q\bar{q}$ mesons and $qqq$ baryons
    \item Directly probe \textbf{how color forces work}
    \item Test our understanding of nonperturbative QCD
\end{itemize}

\vspace{0.4em}
\textbf{\highlight{The Central Question:}}\\
\textit{How do multiple quarks arrange themselves?}
\begin{itemize}
    \item Compact multiquark cluster?
    \item Loosely bound hadronic molecule?
    \item Dynamical mixture of configurations?
\end{itemize}
\end{column}
\begin{column}{0.42\textwidth}
\centering
\begin{tikzpicture}[scale=0.75]
    % Ordinary hadrons
    \node[font=\small\bfseries] at (0,4.2) {Ordinary Hadrons};
    \draw[thick,fill=heavyblue!15] (-1.2,3.2) circle (0.5);
    \fill[heavyblue] (-1.35,3.2) circle (0.12);
    \fill[lightquarkorange] (-1.05,3.2) circle (0.12);
    \node[font=\tiny] at (-1.2,2.5) {$q\bar{q}$};
    \draw[thick,fill=heavyblue!15] (1.2,3.2) circle (0.55);
    \fill[lightquarkorange] (1.0,3.4) circle (0.1);
    \fill[lightquarkorange] (1.4,3.4) circle (0.1);
    \fill[lightquarkorange] (1.2,3.0) circle (0.1);
    \node[font=\tiny] at (1.2,2.5) {$qqq$};

    \draw[->,very thick,heavyblue] (0,2.1) -- (0,1.5);
    \node[font=\tiny] at (0.8,1.8) {go beyond};

    \node[font=\small\bfseries] at (0,1.1) {Exotic Hadrons};
    \draw[thick,fill=orange!15] (-1.2,0) circle (0.6);
    \fill[heavyblue] (-1.4,0.15) circle (0.1);
    \fill[heavyblue] (-1.0,0.15) circle (0.1);
    \fill[lightquarkorange] (-1.3,-0.2) circle (0.08);
    \fill[lightquarkorange] (-1.1,-0.2) circle (0.08);
    \node[font=\tiny] at (-1.2,-0.8) {$QQ\bar{q}\bar{q}$};
    \draw[thick,fill=orange!15] (1.2,0) circle (0.7);
    \fill[heavyblue] (1.0,0.25) circle (0.1);
    \fill[heavyblue] (1.4,0.25) circle (0.1);
    \fill[lightquarkorange] (0.9,-0.15) circle (0.08);
    \fill[lightquarkorange] (1.3,-0.15) circle (0.08);
    \fill[heavyblue] (1.6,-0.1) circle (0.1);
    \node[font=\tiny] at (1.2,-0.9) {$QQqq\bar{Q}$};
\end{tikzpicture}

\vspace{0.3em}
{\small \textit{Understanding exotics $\Rightarrow$\\insights into confinement}}
\end{column}
\end{columns}
\end{frame}

%==============================================================================
% SLIDE: The Multiquark Zoo
%==============================================================================
\begin{frame}{The Multiquark Zoo: Experimental Discoveries}
\begin{columns}[T]
\begin{column}{0.58\textwidth}
\textbf{\highlight{Key Experimental Milestones:}}
\begin{itemize}
    \item \textbf{2003}: $X(3872)$ at Belle --- first exotic candidate
    \item \textbf{2015}: $P_c$ pentaquarks at LHCb ($uudc\bar{c}$)
    \item \textbf{2020}: $T_{4c}(6900)$ at LHCb --- fully heavy $cc\bar{c}\bar{c}$
    \item \textbf{2021}: $T_{cc}(3875)^+$ at LHCb --- doubly charmed $cc\bar{u}\bar{d}$
    \item \textbf{2024}: CMS reports $J^{PC}=2^{++}$ $T_{4c}$ states
\end{itemize}

\vspace{0.5em}
\begin{mybox}
\textbf{This Work:} Study $T_{cc}$, $T_{bb}$, $T_{4c}$, $T_{4b}$, and triply heavy pentaquarks $QQqq\bar{Q}$
\end{mybox}
\end{column}
\begin{column}{0.4\textwidth}
\centering
\begin{tikzpicture}[scale=0.75]
    \node at (0,3.3) {\small\textbf{Compact?}};
    \draw[thick,fill=heavyblue!20] (0,2.4) circle (0.7);
    \fill[heavyblue] (-0.25,2.55) circle (0.12);
    \fill[heavyblue] (0.25,2.55) circle (0.12);
    \fill[lightquarkorange] (-0.25,2.2) circle (0.1);
    \fill[lightquarkorange] (0.25,2.2) circle (0.1);

    \node at (0,1.5) {\small or};

    \node at (0,0.9) {\small\textbf{Molecule?}};
    \draw[thick,fill=heavyblue!10] (-0.5,0.1) circle (0.4);
    \draw[thick,fill=heavyblue!10] (0.5,0.1) circle (0.4);
    \fill[heavyblue] (-0.6,0.2) circle (0.1);
    \fill[lightquarkorange] (-0.4,-0.05) circle (0.08);
    \fill[heavyblue] (0.4,0.2) circle (0.1);
    \fill[lightquarkorange] (0.6,-0.05) circle (0.08);
    \draw[decorate,decoration={snake,amplitude=1.2pt,segment length=4pt}] (-0.1,0.1) -- (0.1,0.1);

    \node at (0,-0.6) {\small\textit{DeepQuark answers this!}};
\end{tikzpicture}
\end{column}
\end{columns}
\end{frame}

%==============================================================================
% SLIDE: Computational Challenges
%==============================================================================
\begin{frame}{Computational Challenges in Multiquark Systems}
\begin{columns}[T]
\begin{column}{0.48\textwidth}
\textbf{\highlight{Exponential Complexity:}}
\begin{itemize}
    \item Wave function dimension scales exponentially
    \item Extra SU(3) color degree of freedom
    \item Multiple quantum numbers: $S$, $I$, $J^{PC}$, color
\end{itemize}

\vspace{0.4em}
\textbf{\highlight{Strong Correlations:}}
\begin{itemize}
    \item Single-particle approximation \stronghighlight{fails}
    \item No shell structure (unlike atoms/nuclei)
    \item Full multi-channel dynamics required
\end{itemize}
\end{column}
\begin{column}{0.48\textwidth}
\textbf{\highlight{Limitations of Existing Methods:}}

\vspace{0.3em}
\textbf{Gaussian Expansion Method (GEM):}
\begin{itemize}
    \item Exponential growth of basis states
    \item Incomplete for 5+ quarks
\end{itemize}

\vspace{0.3em}
\textbf{Diffusion Monte Carlo (DMC):}
\begin{itemize}
    \item Notorious \stronghighlight{sign problem}
    \item Limited for strongly correlated systems
\end{itemize}

\vspace{0.3em}
\textbf{Previous Pentaquark Studies:}
\begin{itemize}
    \item Approximations in spatial configurations
    \item $\Rightarrow$ Unknown systematic errors
\end{itemize}
\end{column}
\end{columns}
\end{frame}

%==============================================================================
% SECTION 2: DEEP LEARNING BACKGROUND
%==============================================================================
\section{Deep Learning Background}
\makesection

%==============================================================================
% SLIDE: What is Machine Learning?
%==============================================================================
\begin{frame}{What is Machine Learning?}
\begin{columns}[T]
\begin{column}{0.55\textwidth}
\textbf{\highlight{Core Idea:}}\\
Learn patterns from data without explicit programming

\vspace{0.4em}
\textbf{\highlight{Types of Learning:}}
\begin{itemize}
    \item \textbf{Supervised:} Learn from labeled examples
    \item \textbf{Unsupervised:} Find hidden structure
    \item \textbf{Variational:} Optimize a target functional\\
    $\Leftarrow$ \stronghighlight{What we use!}
\end{itemize}

\vspace{0.4em}
\begin{mybox}
\textbf{Universal Approximation Theorem:}\\
Neural networks can approximate \textit{any} continuous function to arbitrary accuracy
\end{mybox}
\end{column}
\begin{column}{0.42\textwidth}
\centering
\begin{tikzpicture}[scale=0.9]
    \node[draw,rounded corners,fill=heavyblue!10,minimum width=2.5cm,minimum height=0.8cm] (model) at (0,2) {Neural Network};
    \node[draw,rounded corners,fill=green!10,minimum width=2.5cm,minimum height=0.8cm] (func) at (0,0) {Energy Functional};
    \node[draw,rounded corners,fill=orange!10,minimum width=2.5cm,minimum height=0.8cm] (opt) at (0,-2) {Optimization};

    \draw[->,thick] (model) -- node[right] {$\Psi_\theta$} (func);
    \draw[->,thick] (func) -- node[right] {$E[\Psi_\theta]$} (opt);
    \draw[->,thick] (opt.west) -- ++(-0.8,0) |- node[left,pos=0.25] {update $\theta$} (model.west);

    \node at (0,-3.2) {\small \textbf{Variational Learning}};
\end{tikzpicture}
\end{column}
\end{columns}
\end{frame}

%==============================================================================
% SLIDE: Neural Networks
%==============================================================================
\begin{frame}{Neural Networks: The Basic Building Block}
\begin{columns}[T]
\begin{column}{0.5\textwidth}
\textbf{\highlight{Single Neuron:}}
\begin{equation*}
y = \sigma(\mathbf{W}\mathbf{x} + \mathbf{c})
\end{equation*}
where $\sigma = \tanh$ (activation function)

\vspace{0.4em}
\textbf{\highlight{Physics Analogy: Basis Expansion}}

Traditional wave function:
\begin{equation*}
\Psi(x) = \sum_i c_i \phi_i(x)
\end{equation*}

Neural network:
\begin{equation*}
\Psi(x) = \sum_i w_i \sigma\left(\sum_j W_{ij} x_j + c_j\right)
\end{equation*}
\end{column}
\begin{column}{0.48\textwidth}
\vspace{0.5em}
\begin{mybox}
\textbf{Key difference:}
\begin{itemize}
    \item Basis functions $\phi_i$ are \textit{fixed} in traditional methods
    \item Neural network \textit{learns} the optimal basis!
    \item Adaptive, data-driven representation
\end{itemize}
\end{mybox}
\end{column}
\end{columns}
\end{frame}

%==============================================================================
% SLIDE: VMC Optimization
%==============================================================================
\begin{frame}{Variational Monte Carlo Optimization}
\begin{columns}[T]
\begin{column}{0.52\textwidth}
\textbf{\highlight{Variational Principle:}}
\begin{equation*}
E_{\boldsymbol{\theta}} = \frac{\langle\psi_{\boldsymbol{\theta}}|H|\psi_{\boldsymbol{\theta}}\rangle}{\langle\psi_{\boldsymbol{\theta}}|\psi_{\boldsymbol{\theta}}\rangle} \geq E_0
\end{equation*}
Minimize energy to find ground state!

\vspace{0.4em}
\textbf{\highlight{Monte Carlo Evaluation:}}
\begin{equation*}
E_{\boldsymbol{\theta}} \approx \frac{1}{N}\sum_{n=1}^N E_L(\mathbf{R}_n, \alpha_n)
\end{equation*}
Sample from $|\Psi(\mathbf{R},\alpha)|^2$

\vspace{0.4em}
\textbf{\highlight{Stochastic Reconfiguration:}}
\begin{equation*}
\boldsymbol{\theta}^{i+1} = \boldsymbol{\theta}^i - \eta(S+\epsilon I)^{-1}\nabla E
\end{equation*}
\end{column}
\begin{column}{0.45\textwidth}
\centering
\begin{tikzpicture}[scale=0.85,
    box/.style={draw,rounded corners,minimum width=2.2cm,minimum height=0.7cm,align=center,font=\small}]

    \node[box,fill=heavyblue!20] (sample) at (0,3) {Sample $\mathbf{R},\alpha$\\from $|\Psi|^2$};
    \node[box,fill=orange!20] (compute) at (0,1.5) {Compute\\$E_L$, $\nabla E$};
    \node[box,fill=green!20] (update) at (0,0) {Update $\boldsymbol{\theta}$};
    \node[box,fill=purple!20] (check) at (0,-1.5) {Converged?};

    \draw[->,thick] (sample) -- (compute);
    \draw[->,thick] (compute) -- (update);
    \draw[->,thick] (update) -- (check);
    \draw[->,thick] (check.west) -- ++(-1,0) |- node[left,pos=0.25] {\small No} (sample.west);
    \draw[->,thick] (check.east) -- ++(0.8,0) node[right] {\small Yes};

    \node at (0,4) {\textbf{VMC Loop}};
\end{tikzpicture}
\end{column}
\end{columns}

\vspace{0.2em}
\begin{mybox}
\textbf{Key advantage:} No sign problem (unlike DMC)!
\end{mybox}
\end{frame}

%==============================================================================
% SECTION 3: DEEPQUARK FRAMEWORK
%==============================================================================
\section{DeepQuark Framework}
\makesection

%==============================================================================
% SLIDE: DeepQuark Architecture
%==============================================================================
\begin{frame}{DeepQuark Architecture}
\centering
\includegraphics[width=0.82\textwidth]{arXiv-2506.20555v1/framework.pdf}

\vspace{0.3em}
\textbf{Four types of input:} Spatial coordinates $\mathbf{r}_i$, $|\mathbf{r}_i - \mathbf{r}_j|$ + Color $\alpha_c$ + Spin $\alpha_s$ + Isospin $\alpha_t$
\end{frame}

%==============================================================================
% SLIDE: DeepQuark Wave Function
%==============================================================================
\begin{frame}{DeepQuark Wave Function}
\begin{mybox}
\textbf{Full wave function with built-in symmetries:}
\begin{equation*}
\Psi_{\text{A}}^\pi(\boldsymbol{x}) = (1+\pi\hat{P})\mathcal{A}\left[f_{NN}(\boldsymbol{x})\prod_{i<j}\exp\left(-\frac{r_{ij}^2}{b^2}\right)\right]
\end{equation*}
\end{mybox}

\vspace{0.2em}
\begin{columns}[T]
\begin{column}{0.48\textwidth}
\textbf{\highlight{Key Components:}}
\begin{itemize}
    \item $f_{NN}(\boldsymbol{x})$: Neural network amplitude
    \item $\mathcal{A}[\cdots]$: Antisymmetrization (Fermi-Dirac)
    \item $(1+\pi\hat{P})$: Parity projection ($\pi = \pm 1$)
    \item $e^{-r_{ij}^2/b^2}$: Gaussian boundary ($b \sim 2$--$4$ fm)
\end{itemize}

\vspace{0.3em}
\textbf{\highlight{Input Features:}}
\begin{equation*}
\boldsymbol{x} = (\mathbf{r}_i, |\mathbf{r}_i - \mathbf{r}_j|, \alpha_c, \alpha_s, \alpha_t)
\end{equation*}
\end{column}
\begin{column}{0.48\textwidth}
\textbf{\highlight{Coupled Basis Approach:}}
\begin{itemize}
    \item Color-spin-isospin bases
    \item Example for $QQ\bar{q}\bar{q}$: $\bar{3}_c \otimes 3_c$ and $6_c \otimes \bar{6}_c$
    \item Network learns mixing automatically
\end{itemize}

\vspace{0.3em}
\textbf{\highlight{Why This Works:}}
\begin{itemize}
    \item No \textit{a priori} structure assumption
    \item Same ansatz $\Rightarrow$ molecular or compact
    \item Symmetries built in, not approximated
\end{itemize}
\end{column}
\end{columns}
\end{frame}

%==============================================================================
% SECTION 4: RESULTS
%==============================================================================
\section{Results}
\makesection

%==============================================================================
% SLIDE: Benchmarks
%==============================================================================
\begin{frame}{Benchmarks: DeepQuark Performance}
\begin{columns}[T]
\begin{column}{0.55\textwidth}
\centering
\includegraphics[width=\textwidth]{arXiv-2506.20555v1/Graph_multiquark.pdf}

\vspace{0.1em}
{\small (a) nucleon, (b) $T_{cc}$, (c) $T_{4c}$, (d) pentaquark}
\end{column}
\begin{column}{0.43\textwidth}
\textbf{\highlight{Key Performance Metrics:}}
\begin{itemize}
    \item Matches GEM/DMC to $<0.1$ MeV
    \item Converges in $\sim$1000--3000 iterations
    \item $\sim$1000--3000 parameters (compact!)
\end{itemize}

\vspace{0.3em}
\textbf{\highlight{Unique Capability:}}
\begin{itemize}
    \item Handles \stronghighlight{flux-tube confinement}
    \item GEM cannot do this efficiently!
\end{itemize}

\vspace{0.3em}
\begin{mybox}
\textbf{Same accuracy, handles complex interactions}
\end{mybox}
\end{column}
\end{columns}
\end{frame}

%==============================================================================
% SLIDE: Tcc Tetraquark
%==============================================================================
\begin{frame}{Doubly Charmed Tetraquark $T_{cc}$}
\begin{columns}[T]
\begin{column}{0.45\textwidth}
\centering
\begin{tikzpicture}[scale=0.9]
    \node at (0,3.8) {\textbf{Molecular Structure}};

    \draw[thick,dashed,fill=heavyblue!10] (-1.2,2) ellipse (0.9 and 0.7);
    \draw[thick,dashed,fill=heavyblue!10] (1.2,2) ellipse (0.9 and 0.7);

    \fill[heavyblue] (-1.4,2.2) circle (0.18) node[above,font=\scriptsize] {$c$};
    \fill[lightquarkorange] (-1.0,1.7) circle (0.14) node[below,font=\scriptsize] {$\bar{q}$};
    \fill[heavyblue] (1.0,2.2) circle (0.18) node[above,font=\scriptsize] {$c$};
    \fill[lightquarkorange] (1.4,1.7) circle (0.14) node[below,font=\scriptsize] {$\bar{q}$};

    \draw[decorate,decoration={snake,amplitude=2pt,segment length=5pt},thick] (-0.3,2) -- (0.3,2);

    \node at (-1.2,0.9) {\small $D^{(*)}$};
    \node at (1.2,0.9) {\small $D^{(*)}$};

    \draw[<->,thick,gray] (-1.4,0.2) -- (1.0,0.2);
    \node[gray] at (-0.2,-0.1) {\scriptsize $r_{cc} = 1.24$ fm};
\end{tikzpicture}
\end{column}
\begin{column}{0.52\textwidth}
\textbf{\highlight{Ground State Properties:}}
\begin{itemize}
    \item Binding energy: $\Delta E = \mathbf{-15}$ MeV
    \item \textbf{Color mixing:} $\chi_{\bar{3}\times 3} : \chi_{6\times\bar{6}} = 55\% : 45\%$
    \item Significant mixing of both configurations!
\end{itemize}

\vspace{0.3em}
\textbf{\highlight{RMS Radii (Molecular Structure):}}
\begin{center}
\begin{tabular}{cc}
\toprule
$r_{c\bar{q}}$ & 1.06 fm\\
$r_{cc}$ & 1.24 fm\\
$r_{\bar{q}\bar{q}}$ & 1.41 fm\\
\bottomrule
\end{tabular}
\end{center}

$r_{cc}, r_{\bar{q}\bar{q}} > r_{c\bar{q}}$ $\Rightarrow$ \stronghighlight{Molecular $D^*D$}

\vspace{0.3em}
Consistent with LHCb $T_{cc}(3875)^+$ discovery!
\end{column}
\end{columns}
\end{frame}

%==============================================================================
% SLIDE: Tbb Tetraquark
%==============================================================================
\begin{frame}{Doubly Bottom Tetraquark $T_{bb}$: Compact Diquark}
\begin{columns}[T]
\begin{column}{0.48\textwidth}
\textbf{\highlight{Ground State Properties:}}
\begin{itemize}
    \item Binding energy: $\Delta E = -153$ MeV
    \item \textbf{Color:} $\chi_{\bar{3}\times 3} : \chi_{6\times\bar{6}} = 97\% : 3\%$
    \item Dominated by $\bar{3}_c \otimes 3_c$ configuration!
\end{itemize}

\vspace{0.3em}
\textbf{\highlight{Compact Structure:}}
\begin{center}
\begin{tabular}{cc}
\toprule
$r_{bb}$ & \stronghighlight{0.33 fm} (compact diquark!)\\
$r_{b\bar{q}}$ & 0.69 fm\\
$r_{\bar{q}\bar{q}}$ & 0.78 fm\\
\bottomrule
\end{tabular}
\end{center}

Heavy $bb$ diquark acts like $\bar{3}_c$ antiquark
\end{column}
\begin{column}{0.48\textwidth}
\centering
\begin{tikzpicture}[scale=1.0]
    \node at (-1.5,3.5) {\textbf{$T_{cc}$: Molecular}};
    \draw[thick,dashed,fill=heavyblue!5] (-2.5,1.8) ellipse (0.8 and 0.6);
    \draw[thick,dashed,fill=heavyblue!5] (-0.5,1.8) ellipse (0.8 and 0.6);
    \fill[heavyblue] (-2.7,2) circle (0.12);
    \fill[lightquarkorange] (-2.3,1.6) circle (0.1);
    \fill[heavyblue] (-0.3,2) circle (0.12);
    \fill[lightquarkorange] (-0.7,1.6) circle (0.1);
    \node at (-1.5,0.9) {\small $D^* \leftrightarrow D$};

    \node at (-1.5,-0.3) {\textbf{$T_{bb}$: Compact}};
    \draw[thick,fill=heavyblue!20] (-1.5,-1.8) circle (1.0);
    \draw[thick,fill=heavyblue!40] (-1.5,-1.5) circle (0.35);
    \fill[heavyblue] (-1.62,-1.5) circle (0.1);
    \fill[heavyblue] (-1.38,-1.5) circle (0.1);
    \node at (-1.5,-1.5) [below=0.25cm,font=\tiny] {$bb$ diquark};
    \fill[lightquarkorange] (-2.1,-2.1) circle (0.08);
    \fill[lightquarkorange] (-0.9,-2.1) circle (0.08);
    \node at (-1.5,-2.7) {\small 0.33 fm core};
\end{tikzpicture}
\end{column}
\end{columns}

\vspace{0.2em}
\begin{mybox}
\textbf{Same ansatz} describes both molecular and compact structures!
\end{mybox}
\end{frame}

%==============================================================================
% SLIDE: T4c Laboratory for Confinement
%==============================================================================
\begin{frame}{$T_{4c}$: A Laboratory for Confinement}
\begin{columns}[T]
\begin{column}{0.48\textwidth}
\textbf{\highlight{Why $T_{4c}$ Is Special:}}
\begin{itemize}
    \item \textbf{Pure QCD system} --- no light quark chiral effects
    \item Short-range gluon exchange dominates
    \item Ideal testbed for \stronghighlight{confinement mechanisms}
\end{itemize}

\vspace{0.3em}
\textbf{\highlight{Experimental Motivation:}}
\begin{itemize}
    \item LHCb (2020): $T_{4c}(6900)$ resonance
    \item CMS (2024): Three states with $J^{PC} = 2^{++}$
    \item ATLAS: Confirmation of structures
\end{itemize}

\vspace{0.3em}
\textit{``A clear platform to investigate short-range gluon exchange and confinement''}
\end{column}
\begin{column}{0.48\textwidth}
\textbf{\highlight{Why DeepQuark Can Calculate It:}}

\vspace{0.2em}
\stronghighlight{Challenge:} Flux-tube confinement
\begin{itemize}
    \item Many-body interaction (not pairwise)
    \item ``Computationally intractable'' for GEM
    \item Requires exponentially many basis states
\end{itemize}

\vspace{0.3em}
\textbf{\highlight{DeepQuark Solution:}}
\begin{itemize}
    \item VMC handles complex many-body forces
    \item No basis expansion needed
    \item Monte Carlo sampling is efficient
    \item No sign problem (unlike DMC)
\end{itemize}

\vspace{0.2em}
$\Rightarrow$ Opens door to explore \textbf{confining mechanisms beyond two-body interactions}
\end{column}
\end{columns}
\end{frame}

%==============================================================================
% SLIDE: T4c/T4b Results
%==============================================================================
\begin{frame}{Fully Heavy Tetraquarks: $T_{4c}$ and $T_{4b}$}
\begin{columns}[T]
\begin{column}{0.45\textwidth}
\centering
\begin{tikzpicture}[scale=0.85]
    \node at (0,3.5) {\textbf{No Bound State}};

    \draw[thick,fill=heavyblue!15] (-1.3,1.8) circle (0.7);
    \draw[thick,fill=heavyblue!15] (1.3,1.8) circle (0.7);

    \fill[heavyblue] (-1.5,2) circle (0.15);
    \fill[heavyblue] (-1.1,1.6) circle (0.15);
    \fill[heavyblue] (1.1,2) circle (0.15);
    \fill[heavyblue] (1.5,1.6) circle (0.15);

    \node at (-1.3,0.8) {\small $\eta_c/J/\psi$};
    \node at (1.3,0.8) {\small $\eta_c/J/\psi$};

    \draw[dashed,thick,red] (-2.2,0.2) -- (2.2,0.2) node[right,font=\small] {threshold};
    \draw[thick,blue] (-2.2,-0.3) -- (2.2,-0.3) node[right,font=\small] {$E_{DQ}$};

    \node at (0,-0.8) {\small Energy \textbf{above} threshold};
\end{tikzpicture}
\end{column}
\begin{column}{0.52\textwidth}
\textbf{\highlight{Results:}}
\begin{center}
\begin{tabular}{lcc}
\toprule
System & $S^P$ & Bound?\\
\midrule
$cc\bar{c}\bar{c}$ & $0^+, 1^+, 2^+$ & \stronghighlight{No}\\
$bb\bar{b}\bar{b}$ & $0^+, 1^+, 2^+$ & \stronghighlight{No}\\
\bottomrule
\end{tabular}
\end{center}

\vspace{0.2em}
\textbf{Color proportion:} $\chi_{\bar{3}\times 3} : \chi_{6\times\bar{6}} \approx 1:2$\\
$\Rightarrow$ Consistent with meson-meson scattering

\vspace{0.3em}
\textbf{\highlight{Experimental context:}}
\begin{itemize}
    \item LHCb (2020): $T_{4c}(6900)$ resonance
    \item CMS (2025): Three $T_{4c}$ with $J^{PC} = 2^{++}$
\end{itemize}

\vspace{0.2em}
$\Rightarrow$ Observed structures are \stronghighlight{resonances}, not bound states
\end{column}
\end{columns}
\end{frame}

%==============================================================================
% SLIDE: Novel S=5/2 Pentaquarks
%==============================================================================
\begin{frame}{Novel Prediction: Bound $S = 5/2$ Pentaquarks}
\begin{columns}[T]
\begin{column}{0.45\textwidth}
\centering
\begin{tikzpicture}[scale=0.85]
    \node at (0,3.8) {\textbf{Molecular Structure}};

    \draw[thick,fill=heavyblue!30] (-0.8,1.8) circle (0.9);
    \fill[heavyblue] (-1.0,2.1) circle (0.12);
    \fill[heavyblue] (-0.6,2.1) circle (0.12);
    \fill[lightquarkorange] (-1.1,1.5) circle (0.1);
    \fill[lightquarkorange] (-0.5,1.5) circle (0.1);
    \node at (-0.8,0.6) {\small $\Xi_{cc}^*$};

    \draw[thick,dashed,fill=heavyblue!15] (1.5,1.8) circle (0.6);
    \fill[heavyblue] (1.3,1.9) circle (0.12);
    \fill[lightquarkorange] (1.7,1.6) circle (0.1);
    \node at (1.5,0.9) {\small $\bar{D}^*$};

    \draw[decorate,decoration={snake,amplitude=2pt,segment length=5pt},thick] (0.1,1.8) -- (0.9,1.8);

    \draw[<->,thick,gray] (-0.8,0.0) -- (1.5,0.0);
    \node[gray] at (0.35,-0.4) {\scriptsize $r_{c\bar{c}} = 1.73$ fm};

    \node[draw,rounded corners,fill=green!20] at (0.3,-1.2) {\small $\Delta E = -3$ MeV};
\end{tikzpicture}
\end{column}
\begin{column}{0.52\textwidth}
\textbf{\highlight{Why $S = 5/2$ is special:}}\\
S-wave $S=\frac{3}{2}$ isoscalar baryon is \stronghighlight{forbidden} by Fermi statistics!

\vspace{0.2em}
$\Rightarrow$ Lowest threshold: $\bar{D}^*\Xi_{cc}^*$ (or $B^*\Xi_{bb}^*$)

\vspace{0.3em}
\textbf{\highlight{Bound states found:}}
\begin{center}
\begin{tabular}{lcc}
\toprule
State & Mass & $\Delta E$\\
\midrule
$P_{cc\bar{c}}(5715)$ & 5715 MeV & $\mathbf{-3}$ MeV\\
$P_{bb\bar{b}}(15569)$ & 15569 MeV & $\mathbf{-14}$ MeV\\
\bottomrule
\end{tabular}
\end{center}

\vspace{0.2em}
\textbf{Structure:} Molecular $\bar{D}^*\Xi_{cc}^*$
\begin{itemize}
    \item $r_{cc} = 0.50$ fm (compact $\Xi_{cc}^*$)
    \item $r_{c\bar{c}} = 1.73$ fm (large separation)
\end{itemize}
\end{column}
\end{columns}
\end{frame}

%==============================================================================
% SECTION 5: CONCLUSIONS
%==============================================================================
\section{Conclusions}
\makesection

%==============================================================================
% SLIDE: Summary
%==============================================================================
\begin{frame}{Summary}
\textbf{\highlight{DeepQuark:}} First DNN-based VMC for multiquark bound states

\vspace{0.4em}
\begin{columns}[T]
\begin{column}{0.48\textwidth}
\textbf{Method Achievements:}
\begin{itemize}
    \item Novel coupled color-spin-isospin bases
    \item Unbiased compact \& molecular description
    \item Handles flux-tube confinement efficiently
    \item Competitive with GEM and DMC
    \item Scalable to larger systems
\end{itemize}
\end{column}
\begin{column}{0.48\textwidth}
\textbf{Physics Results:}
\begin{itemize}
    \item \stronghighlight{$T_{cc}$}: Molecular, $\Delta E = -15$ MeV
    \item \stronghighlight{$T_{bb}$}: Compact diquark, $\Delta E = -153$ MeV
    \item \stronghighlight{$T_{4c}, T_{4b}$}: No bound states
    \item \textbf{New predictions:}
    \begin{itemize}
        \item $P_{cc\bar{c}}(5715)$: $-3$ MeV
        \item $P_{bb\bar{b}}(15569)$: $-14$ MeV
    \end{itemize}
\end{itemize}
\end{column}
\end{columns}

\vspace{0.3em}
\begin{mybox}
\textbf{Experimental search:} $P_{cc\bar{c}}(5715)$ in D-wave $J/\psi\,\Lambda_c$ at LHCb
\end{mybox}
\end{frame}

%==============================================================================
% SLIDE: Outlook
%==============================================================================
\begin{frame}{Outlook: Toward Understanding Confinement}
\begin{columns}[T]
\begin{column}{0.48\textwidth}
\textbf{\highlight{Immediate Physics Goals:}}
\begin{itemize}
    \item Other pentaquark systems ($P_c$, $P_b$)
    \item Hexaquarks (6 quarks, like $d^*$)
    \item Excited states and resonances
\end{itemize}

\vspace{0.3em}
\textbf{\highlight{Probing Confinement:}}
\begin{itemize}
    \item Flux-tube vs.\ pairwise confinement
    \item Many-body color interactions
    \item Connection to lattice QCD
    \item \textit{Which mechanism governs multiquarks?}
\end{itemize}

\vspace{0.2em}
DeepQuark is \stronghighlight{uniquely positioned} to explore these questions!
\end{column}
\begin{column}{0.48\textwidth}
\textbf{\highlight{Experimental Synergy:}}
\begin{itemize}
    \item LHCb, CMS, ATLAS: more exotics coming
    \item BESIII: charm sector
    \item Belle II: B physics
    \item Predictions guide searches
\end{itemize}

\vspace{0.3em}
\textbf{\highlight{The Big Picture:}}
\begin{itemize}
    \item Exotics $\Rightarrow$ probe nonperturbative QCD
    \item Structure reveals color dynamics
    \item DeepQuark: first-principles predictions
    \item Deep learning enables previously intractable calculations
\end{itemize}
\end{column}
\end{columns}
\end{frame}

%==============================================================================
% SLIDE: Acknowledgments
%==============================================================================
\begin{frame}{Acknowledgments}
\begin{columns}[T]
\begin{column}{0.55\textwidth}
\textbf{\highlight{Collaborators:}}
\begin{itemize}
    \item Wei-Lin Wu (Peking University)
    \item Shi-Lin Zhu (Peking University)
\end{itemize}

\vspace{0.4em}
\textbf{\highlight{Helpful Discussions:}}
\begin{itemize}
    \item Yao Ma, Yan-Ke Chen, Liang-Zhen Wen
    \item Yilong Yang, Pengwei Zhao
\end{itemize}

\vspace{0.4em}
\textbf{\highlight{Funding:}}
\begin{itemize}
    \item NSFC (No.\ 12475137)
    \item ERC NuclearTheory (Grant No.\ 885150)
\end{itemize}

\vspace{0.3em}
\textbf{Computing:} HPC Platform of Peking University\\
\textbf{Software:} NetKet package
\end{column}
\begin{column}{0.4\textwidth}
\centering
\vspace{1cm}
{\Huge\color{themecolorfg}\textbf{Thank you!}}

\vspace{1cm}
\textbf{Paper:}\\
arXiv:2506.20555

\vspace{0.5cm}
\textbf{Contact:}\\
lu.meng@rub.de
\end{column}
\end{columns}
\end{frame}

\end{document}
