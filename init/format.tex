% Beamer 模板设置
\setbeamertemplate{navigation symbols}{}
\setbeamertemplate{footline}[page number]
\setbeamertemplate{section in toc}[sections numbered]

% Itemize 格式设置
\setbeamertemplate{itemize item}[circle] % 项目符号设置为圆形
\setbeamertemplate{itemize subitem}{$\circ$} % 子项目符号设置为圆形
\setbeamertemplate{itemize subsubitem}[circle] % 子子项目符号设置为圆形

% Enumerate 格式设置
\setbeamertemplate{enumerate subitem}{\alph{enumii}.} % 子项目符号设置为小写字母
\setbeamertemplate{enumerate subsubitem}{\roman{enumiii}.} % 子子项目符号设置为罗马数字

% 颜色设置
\setbeamercolor{frametitle}{bg=, fg=themecolorfg}
\setbeamercolor{block title}{bg=themecolorbg!8, fg=themecolorfg}
\setbeamercolor{description item}{fg=themecolorfg}
\setbeamercolor{section in toc}{fg=themecolorfg, bg=}
\setbeamercolor{subsection in toc}{fg=black!80}
\setbeamercolor{caption name}{fg=themecolorfg}
\setbeamercolor{bibliography entry author}{fg=black}
\setbeamercolor{bibliography entry title}{fg=black}
\setbeamercolor{bibliography entry journal}{fg=black}
\setbeamercolor{bibliography entry note}{fg=black}
\setbeamercolor{bibliography item}{fg=themecolorfg}
\setbeamercolor{itemize item}{fg=themecolorbg} % 项目符号颜色设置
\setbeamercolor{itemize subitem}{fg=themecolorbg} % 子项目符号颜色设置
\setbeamercolor{itemize subsubitem}{fg=themecolorbg} % 子子项目符号颜色设置
\setbeamercolor{enumerate item}{fg=themecolorfg} % 枚举项颜色设置
\setbeamercolor{enumerate subitem}{fg=themecolorfg} % 枚举子项颜色设置
\setbeamercolor{enumerate subsubitem}{fg=themecolorfg} % 枚举子子项颜色设置

% ============================================
% 语言模式检测 (en 或 cn)
% ============================================
\newif\iflangcn
\ifdefined\langmode
  % 使用 \pdfstrcmp 或字符串比较 (更可靠)
  \makeatletter
  \def\@tempa{cn}
  \ifx\langmode\@tempa
    \langcntrue
  \else
    \langcnfalse
  \fi
  \makeatother
\else
  % 默认为中文模式
  \langcntrue
\fi

% ============================================
% 字体设置(根据语言模式)
% 定义全局字体变量,修改这里即可改变所有字体
% ============================================
\iflangcn
  % 中文模式:使用中文字体
  % 可选: \kaishu, \songti, \heiti, \fangsong, \yahei, \lishu 等
  \newcommand{\slidefont}{\kaishu}  % 标题字体 (frametitle, block title 等)
  \newcommand{\bodyfont}{\kaishu}     % 正文字体
    \AtBeginDocument{\bodyfont}

\else
  % 英文模式:使用默认字体
  \newcommand{\slidefont}{}  % 标题字体
  \newcommand{\bodyfont}{}  % 正文字体
\fi

% 字体设置 
\setbeamerfont{frametitle}{size=\Large,family=\slidefont}
\setbeamerfont{framesubtitle}{size=\normalsize,family=\slidefont}
\setbeamerfont{block title}{size=\large,family=\slidefont}
\setbeamerfont{section in toc}{size=\large,family=\slidefont}
\setbeamerfont{subsection in toc}{size=\normalsize,family=\slidefont}
\setbeamerfont{caption name}{size=\normalsize,family=\slidefont}
\setbeamerfont{caption}{size=\normalsize,family=\slidefont}
\setbeamerfont{footnote}{size=\tiny}
\setbeamerfont{description item}{size=\normalsize,family=\slidefont}

% 页脚模板设置
\setbeamertemplate{footline}{%
  \begin{beamercolorbox}[wd=\paperwidth, ht=2.25ex, dp=1ex, leftskip=0.5cm, rightskip=0.5cm]{title in head/foot}%
    \hfill{\textcolor{black!70}{\scriptsize \insertframenumber/\inserttotalframenumber}}\hspace*{1mm}\vspace*{2mm}
  \end{beamercolorbox}%
}

% 自定义标题模板 (使用 \slidefont 而不是 \bf)
\defbeamertemplate*{frametitle}{}[1][]{
  \nointerlineskip%
   \vspace*{-7mm}
  \hspace*{-1mm}
  \begin{beamercolorbox}[sep=0.3cm, wd=\paperwidth]{frametitle}
  \begin{center}
    {\bf\usebeamerfont{frametitle}\insertframetitle}
    {\usebeamerfont{framesubtitle}\color{black!80}\insertframesubtitle}
  \end{center}
       \vskip-4.5ex
    \hfill
    \raisebox{-3mm}{\includegraphics[width=20mm]{figs/logo_wb}}
           \vskip-1.5ex
    \begin{tikzpicture}[remember picture, overlay]
      \draw[line width=1pt, color=themecolorbg] (-14mm,0) -- (\paperwidth-28mm,0);
    \end{tikzpicture}
  \end{beamercolorbox}
  \vspace*{-7mm}
}



% 设置段落间距
\setlength{\parskip}{0.5em}

% 参考文献
\bibliography{reference.bib}
\normalem

% 数学字体主题设置
\usefonttheme[onlymath]{serif}

% 表格环境设置
\AtBeginEnvironment{table}{\setlength{\parskip}{0em}}


% ------------------

\newmdenv[
    hidealllines=true,
    leftline=true,
    linewidth=1pt,
    linecolor=darkgray,
    fontcolor=darkgray,
    innertopmargin=1pt,
    innerbottommargin=1pt,
    font=\bodyfont\normalsize,
]{myquote}

\newmdenv[
    hidealllines=true,
    linewidth=0pt,
    fontcolor=darkgray,
    font=\bodyfont\small,
]{myderive}

\newmdenv[
userdefinedwidth=0cm,
leftmargin=2cm,
linecolor=themecolorbg,
innerleftmargin=5pt,
 backgroundcolor=themecolorbg!10, % 设置填充颜色(此处为蓝色,20%透明度
align=center]{mybox}


\lstset{
    basicstyle=\ttfamily\color{stronghighlightcolor},
}

\lstdefinestyle{python}{
    language=Python,
    basicstyle=\normalsize\ttfamily,
    keywordstyle=\color{themecolorbg},
    stringstyle=\color{themered!150!black},
    commentstyle=\color{black!50},
    showstringspaces=false,
    backgroundcolor=\color{white},
    frame=single,
    framerule=0.25pt,
    framesep=3pt,
    rulecolor=\color{themecolorbg},
    numbers=left,
    numberstyle=\tiny\color{darkgray},
    numbersep=5pt,
    breaklines=true,
    escapeinside={(*@}{@*)}
}

\lstdefinestyle{latex}{
    language=[LaTeX]TeX,
    basicstyle=\normalsize\ttfamily,
    keywordstyle=\color{themecolorbg},
    stringstyle=\color{themered!150!black},
    commentstyle=\color{black!50},
    showstringspaces=false,
    backgroundcolor=\color{white},
    frame=single,
    framerule=0.25pt,
    framesep=3pt,
    rulecolor=\color{themecolorbg},
    numbers=left,
    numberstyle=\tiny\color{darkgray},
    numbersep=5pt,
    breaklines=true,
    escapeinside={(*@}{@*)}
}

% ------------------------

\newcommand{\makesection}[1][0.4]{
    \begin{frame}[plain]
        \centering
        \vspace*{7mm}
        \begin{tabular}{l}
            {\color{themecolorfg}
            \huge\slidefont
            \insertsectionnumber.\ \insertsection}\\
            \begin{tikzpicture}
                \fill[black!30] (0,0) rectangle (#1\textwidth, 1pt);
                \fill[themecolorfg] (0,0) rectangle (#1\textwidth/2, 1pt);
            \end{tikzpicture}
        \end{tabular}
    \end{frame}
}


\newcommand{\titlecolorbox}[2]{
    \hfill
\begin{tikzpicture}[remember picture, overlay]
  \node[
    anchor=north east,
    xshift=-0.5cm,   % 向左偏移
    yshift=-0.3cm    % 向下偏移
  ] at (current page.north east)
  {\includegraphics[width=3cm]{figs/logo_cl.pdf}};
\end{tikzpicture}


    \centering

     \vspace{5mm}
       \begin{tcolorbox}[colback=white!0, colframe=white!0,
    boxrule=0pt,
    toprule=0pt,
    leftrule=0pt,
    rightrule=0pt, arc=0mm,  width=\paperwidth,
   enlarge left by=-1in-\oddsidemargin,
    enlarge right by=-1in-\oddsidemargin ]
        \color{darkgray} \normalsize\slidefont
        #2
    \end{tcolorbox}
       \vspace{-5mm}
    \begin{tcolorbox}[colback=themecolorbg, colframe=themeyellow,
    boxrule=5pt,
    toprule=0pt,
    leftrule=0pt,
    rightrule=0pt, arc=0mm,  width=\paperwidth,
   enlarge left by=-1in-\oddsidemargin,
    enlarge right by=-1in-\oddsidemargin ]
        \color{white} \Huge\slidefont\centering
        #1
    \end{tcolorbox}
}

\newcommand{\highlight}[1]{\textcolor{themecolorbg}{#1}}
\newcommand{\ulhighlight}[1]{\textcolor{themecolorbg}{\color{themecolorbg}\uline{#1}}}
\newcommand{\stronghighlight}[1]{\textcolor{stronghighlightcolor}{#1}}
\newcommand{\ulstronghighlight}[1]{\textcolor{stronghighlightcolor}{\color{stronghighlightcolor}\uline{#1}}}
